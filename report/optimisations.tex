\section{Further optimisations}\label{sec:further-optimisations}

\subsection{Architectural optimisations}

In order to optimise the circuit, a minimum area target was defined.

To achieve this, some strategies were followed, as is explained below.

\subsubsection*{Coding}

Below are listed the performed optimisations:
\begin{itemize}
	\item Definition of \texttt{seqmultNM} blocks with \texttt{N} (multiplier) having the most bits and \texttt{M} (multiplicand) having the least bits. This increases the number of cycles needed for computation (but still manages to meet timing constraints), but allows for decreases in area
	\item Elimination of all duplicate \texttt{reg} and \texttt{wire} instantiations
\end{itemize}

Further optimisation was also attempted. However, some measures didn't translate into additional optimisation gains. The attempted procedures are listed below:
\begin{itemize}
	\item Condensation of all division and multiplication scenarios in \texttt{seqmultNM\_sat} by using a \emph{wire} assignment before comparing the final value
	\item Utilisation of \texttt{else ... if} statements when in the presence of mutually exclusive statements (e.g. \texttt{x} versus \texttt{!x})
	\item Utilisation of registers to save all previous inputs from multipliers, in order to lower power consumption in case the results remained unaltered (removed due to increase in area size)
	\item Utilisation of two-step sum in \texttt{block\_192kHz} (i.e. by performing two sums with two operands instead of a single sum with three operators
\end{itemize}

\subsubsection*{Synthesis}
After validation of the final design, synthesis optimisation was achieved by means of editing the \texttt{Optimisation Goal} and \texttt{Optimisation Effort} parameters under the \texttt{Synthesis - XST > Process Properties} options.

Maximum optimisation was achieved using the values listed below. The resulting device utilisation statistics are presented in \autoref{table:device-utilisation}.

\begin{table}[!h]
\center
\begin{tabular}{|p{6cm}|p{2cm}|}
\hline
\textbf{Optimisation Goal}   & Speed  \\ \hline
\textbf{Optimisation Effort} & Normal \\ \hline
\end{tabular}
\end{table}

\subsection{Resource occupancy}

The results for the resource occupancy of the device upon synthesis follow below.

%Having the clock frequency fixed to \emph{insert\_freq\_here}, 
\begin{table}[!h]
\center
\begin{tabular}{|l|R{2.25cm}|R{2.25cm}|R{2.25cm}|}
\hline
\rowcolor[HTML]{FFFF99} 
\textbf{Logic utilisation}        & \multicolumn{1}{l|}{\cellcolor[HTML]{FFFF99}\textbf{Used}} & \multicolumn{1}{l|}{\cellcolor[HTML]{FFFF99}\textbf{Available}} & \multicolumn{1}{l|}{\cellcolor[HTML]{FFFF99}\textbf{Utilisation}} \\ \hline
Number of Slice Registers         & 1732                                                       & 54576                                                           & 3\%                                                               \\ \hline
Number of Slice LUTs              & 1319                                                       & 27288                                                           & 5\%                                                               \\ \hline
Number of fully used LUT-FF pairs & 898                                                       & 1778                                                            & 50\%                                                              \\ \hline
Number of bonded IOBs             & 78                                                         & 218                                                             & 35\%                                                              \\ \hline
Number of Block RAM/FIFO          & 0                                                          & 116                                                             & 0\%                                                               \\ \hline
\end{tabular}
\caption{Device utilisation results for the final optimised synthesis}\label{table:device-utilisation}
\end{table}

\subsection{Additional changes}

Besides this, a change was performed to the FM stereo data generation data generation script (\texttt{fmstereo\_gensimdata.m}). An imprecision was detected in lines 135 and 137, in which the generated data, when saturating to a negative value, would exceed the representation range of the hardware, a wrong value. The code was changed so as to saturate to the most negative value possible to represent by its number of bits.
