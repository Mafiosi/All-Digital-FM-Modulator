\section{Control path}\label{sec:control-path}

In this section the control path will be explained, followed by the implementation of both \texttt{block\_48kHz} and \texttt{block\_192kHz} blocks. These are an application of that same algorithm.

The multiplication block is essential for the correct functioning of the system. As such, the control path was designed around its requirements, as detailed below.

\begin{lstlisting}[caption={Description of the control path}, label={lst:control}]
	* Start phase:
		- The inputs are continually updated in accordance with each multiplier
		- The start flag of the multiplier is enabled
	* Multiplier phase:
		- The control flag is enabled when the ready flag is low
	* Final phase:
		- Results are signalled as ready when both control and ready flags are enabled
		- Upon acknowledging the valid results, the control flag is set to low
\end{lstlisting}

\subsection{block\_48kHz}

For this block, both inputs are being added and subtracted continuously by means of an \emph{assign} condition. Any possible overflow is avoided by discarding the LSB as a result of a previous sum which preserves any overflow information.

Due to requirements of the \texttt{seqmultNM} blocks, both \texttt{Ks} and \texttt{Kd} gains are extended by a signal bit, so as to be interpreted as signed (and positive) variables.

The multiplier is enabled using the signal \texttt{clocken\_48}, which is active during a single clock cycle, with a periodicity of 48 kHz. After this, the strategy in \autoref{lst:control} is used, and its output is directed to the inputs of the \texttt{interpol4x} blocks. Using the \texttt{seqmultNM\_sat} described under \autoref{sec:further-optimisations}, the outputs of both multipliers are shifted by 3 bits to the right.

\subsection{block\_192kHz}

For this block, a similar implementation is performed. The gains \texttt{Kp} and \texttt{Kf} are extended by a signal bit. The output of both \texttt{dds} blocks is placed in a 8-bit register. 

The multipliers at the left of the sum block (see \autoref{fig:overview}) are enabled using the signal \texttt{clocken\_192}, which is active during a single clock cycle, with a periodicity of 192 kHz. Besides this, a condition is added, which signals that all calculations are finished within the block. For the multiplication with the 38 kHz sine wave, the result is affected by an 8-bit right shift. For the multiplication with the 19 kHz sine wave, the result is affected by an 6-bit left shift. Both outputs are rescaled using the \texttt{seqmultNM\_sat} block.

A single \emph{assign} statement is used for the summation of all three values before the rightmost multiplication. When both multiplications before this sumation finishes, as described under \autoref{lst:control}, the results are acknowledged. This disables the flags of the multipliers on the left of the summation, and enables the rightmost multiplication. Its output already counts for a 4-bit shift. A control flag is enabled, which states that all calculations are finished, and its values are loaded into a registered output with a periodicity of 192 kHz.
